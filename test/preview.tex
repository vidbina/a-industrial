\documentclass[a4paper]{article}
\usepackage{fontspec, xunicode, xltxtra}
\setlength\parindent{0pt}
%\setmainfont[RawFeature={+liga,+kern,+ss01,+onum},Scale=2.0]{A Industrial Black}
%\setmainfont[Scale=2.0]{A Industrial Black}
\newfontfamily{\ssdefault}[Scale=2.0]{A Industrial Black}
\newfontfamily{\ssone}[StylisticSet=1,Scale=2.0]{A Industrial Black}
\newfontfamily{\sstwo}[StylisticSet=2,Scale=2.0]{A Industrial Black}
\newfontfamily{\ssthree}[StylisticSet=3,Scale=2.0]{A Industrial Black}
\newfontfamily{\ssfour}[StylisticSet=4,Scale=2.0]{A Industrial Black}
\newfontfamily{\ssfive}[StylisticSet=5,Scale=2.0]{A Industrial Black}
\begin{document}
\title{\input{../NAME}}
\author{}
\maketitle
\begin{abstract}
  The different stylistic sets of the font are presented in this preview in order to provide an indication of the alternative glyphs available for use.
  Considering that Industrial classifies as a Display type, there will be plenty of occasions where the default glyphs fail to produce wordshapes that allow for quick reading.
\end{abstract}

%\fontsize{24pt}{36pt}\selectfont

\newpage
\section{default}
{\ssdefault \input{content/nato.tex}
\input{content/others.tex}
41+27=68 \\
39/3=13 \\

hamburgerfontsiv \\
difficult waffles \\
handgloves \\
asabina

}

\newpage
\section{ss01}
{\ssone \input{content/nato.tex}
\input{content/others.tex}
41+27=68 \\
39/3=13 \\

hamburgerfontsiv \\
difficult waffles \\
handgloves \\
asabina

}

\newpage
\section{ss02}
{\sstwo \input{content/nato.tex}
\input{content/others.tex}
41+27=68 \\
39/3=13 \\

hamburgerfontsiv \\
difficult waffles \\
handgloves \\
asabina

}

\newpage
\section{ss03}
{\ssthree \input{content/nato.tex}
\input{content/others.tex}
41+27=68 \\
39/3=13 \\

hamburgerfontsiv \\
difficult waffles \\
handgloves \\
asabina

}

\newpage
\section{ss04}
{\ssfour \input{content/nato.tex}
\input{content/others.tex}
41+27=68 \\
39/3=13 \\

hamburgerfontsiv \\
difficult waffles \\
handgloves \\
asabina

}


\newpage
\section{Cases}
The following sections highlight some challenges with the typeface and proposes
some solutions which have been made available through alternate glyphs through
a number of stylistic sets.

\subsection{e}
The word {\ssdefault aeon} ("aeon") is hardly legible because the characters
are all very rectangular by shape as used to represent the character "o"
{\ssdefault o}. In some cases the shape of the word assists minimizing
ambiguity, a feat we can not really rely on in this case.

By exchanging the "e" for one of the alternatives as in {\ssdefault a\ssone{e}on} where "e" has
been replaced by an alternate from the first stylistic set (ss01), one may minimize
the ambiguity, however; due to the contraint of the grid the "familiar"
wordshape is distrupted, possibly complicating legibility.

\subsection{x}
{\ssdefault maximum}

\end{document}
