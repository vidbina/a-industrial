\documentclass[a4paper]{article}
\usepackage{fontspec, xunicode, xltxtra}
\setlength\parindent{0pt}
%\setmainfont[RawFeature={+liga,+kern,+ss01,+onum},Scale=2.0]{A Industrial Black}
%\setmainfont[Scale=2.0]{A Industrial Black}
\newfontfamily{\ssdefault}[Scale=2.0]{A Industrial Black}
\newfontfamily{\ssone}[StylisticSet=1,Scale=2.0]{A Industrial Black}
\newfontfamily{\sstwo}[StylisticSet=2,Scale=2.0]{A Industrial Black}
\newfontfamily{\ssthree}[StylisticSet=3,Scale=2.0]{A Industrial Black}
\newfontfamily{\ssfour}[StylisticSet=4,Scale=2.0]{A Industrial Black}
\begin{document}
\title{\input{../NAME}}
\author{}
\maketitle
\begin{abstract}
  The different stylistic sets of the font are presented in this preview in order to provide an indication of the alternative glyphs available for use.
  Considering that Industrial classifies as a Display type, there will be plenty of occasions where the default glyphs fail to produce wordshapes that allow for quick reading.
\end{abstract}

%\fontsize{24pt}{36pt}\selectfont

\newpage
\section{default}
{\ssdefault alpha bravo charlie \\
delta echo foxtrot \\
golf hotel india \\
juliet kilo lima \\
mike november oscar \\
papa quebec romeo \\
sierra tango uniform \\
victor whiskey xray \\
yankee zulu \\

\#hashtag \\
@someone \\

41+27=68 \\
39/3=13 \\

}

\newpage
\section{ss01}
{\ssone alpha bravo charlie \\
delta echo foxtrot \\
golf hotel india \\
juliet kilo lima \\
mike november oscar \\
papa quebec romeo \\
sierra tango uniform \\
victor whiskey xray \\
yankee zulu \\

\#hashtag \\
@someone \\

41+27=68 \\
39/3=13 \\

}

\newpage
\section{ss02}
{\sstwo alpha bravo charlie \\
delta echo foxtrot \\
golf hotel india \\
juliet kilo lima \\
mike november oscar \\
papa quebec romeo \\
sierra tango uniform \\
victor whiskey xray \\
yankee zulu \\

\#hashtag \\
@someone \\

41+27=68 \\
39/3=13 \\

}

\newpage
\section{ss03}
{\ssthree alpha bravo charlie \\
delta echo foxtrot \\
golf hotel india \\
juliet kilo lima \\
mike november oscar \\
papa quebec romeo \\
sierra tango uniform \\
victor whiskey xray \\
yankee zulu \\

\#hashtag \\
@someone \\

41+27=68 \\
39/3=13 \\

}

\newpage
\section{ss04}
{\ssfour alpha bravo charlie \\
delta echo foxtrot \\
golf hotel india \\
juliet kilo lima \\
mike november oscar \\
papa quebec romeo \\
sierra tango uniform \\
victor whiskey xray \\
yankee zulu \\

\#hashtag \\
@someone \\

41+27=68 \\
39/3=13 \\

}


\newpage
\section{Cases}
The word {\ssdefault aeon} ("aeon") is hardly legible because the characters are all very
similar to a simple square as used to represent the character "o" {\ssdefault o}. In some
cases the shape of the word assists minimizing ambiguity, a feat we can not really rely on
in this case.

By exchanging the "e" for one of the alternatives as in {\ssdefault a\ssone{e}on}, one may
minimize the ambiguity, however; due to the strong adherence to the grid this is not achieved
by maintaining a "familiar" wordshape.

\end{document}
