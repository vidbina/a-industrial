\subsection{e}
The word {\ssdefault aeon} (``aeon") is hardly legible because the characters
are all very rectangular by shape as used to represent the character ``o"
depicted by default as {\ssdefault o}. In some cases the shape of the word aids
in minimizing ambiguity, a feat we can not really rely on in this case.

\altfig{e}{1}

By exchanging the ``e" for one of the alternatives as in
{\ssdefault a\ssone{e}on} where ``e" has been replaced by an alternate from the
first stylistic set (ss01), one may minimize the ambiguity, however; due to the
contraint of the grid the ``familiar" wordshape is distrupted, possibly
complicating legibility.

\subsubsection{default {\ssdefault e}}
In order to maintain the horizontal flow of the typeface's design, the terminal
of the ``e" is implied. Due to this implied feature, certain use-cases lead to
less-legible outcomes as evident in the depiction of the word
\mbox{{\ssdefault eat} (eat)} which is easily misinterpreted as ``oat".

Other words such as
\mbox{{\ssdefault basket} (basket)},
\mbox{{\ssdefault television} (television)} and
\mbox{{\ssdefault elephant} (elephant)}
are relatively easily legible due to the distinctive and familiar wordforms.

\subsubsection{{\ssone e}.ss01}

In cases where the a more pronounced terminal for the letter ``e" is required,
one may op for the use of thee ss01 alternative glyph which, due to the
restrictions imposed by the grid, will break the horizontal flow of the
typeface as visible through the representation of
\mbox{{\ssdefault {\ssone e}at} (eat)}.
