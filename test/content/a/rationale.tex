\subsection{a}

The word ``area" presented by default as {\ssdefault area} is hardly
distinguishable from {\ssdefault oreo} (``oreo") which is why a typesetter may
decide to use one of the stylistic alternates for ``a" to produce
{\ssdefault{\ssone a}re{\ssone a}} (ss01),
{\ssdefault{\ssthree a}re{\ssthree a}} (ss03) or
{\ssdefault{\ssfour a}re{\ssfour a}} (ss04).

The ss02 and ss05 alternatives look more like ``dred" or ``drod", respectively
{\ssdefault{\sstwo a}re{\sstwo a}} (ss02) or
{\ssdefault{\ssfive a}re{\ssfive a}} (ss05)
which fails to improve legibility in this particular case.


The following glyphs are available for ``a":
\begin{enumerate}
  \item {\ssdefault a} --- harmonizes best with the overal design language of
    the typeface but fails in avoiding ambiguity between letterforms when this
    particular glyph is surrounded by letters with a height close to the
    x-height and an overal rectangular profile as evident in the presentation
    of the word \mbox{{\ssdefault area} (area)}, which may easily be confused
    for ``oreo", ``drod'' or ``dred" while conversely
    \mbox{{\ssdefault transport} (transport)} or
    \mbox{{\ssdefault father} (father)} remain legible
  \item {\ssone a} (ss01): provides a horizontal stem that may at times be
    misread as ``o" as in \mbox{\ssdefault m{\ssone aa}g} which represents
    ``maag" but could be misinterpreted as ``moog"; this glyph does aid in
    maintaining the horizontal flow of a disinctive wordforms as evident in
    %\mbox{{\ssdefault{\ssone a}rea} (area)},
    \mbox{{\ssdefault d{\ssone a}ddy} (daddy)},
    \mbox{{\ssdefault {\ssone a}lphabet} (alphabet)},
    \mbox{{\ssdefault {\ssone a}lph{\ssone a}bet} (alphabet)},
    \mbox{{\ssdefault {\ssone a}llah} (allah)} and
    \mbox{{\ssdefault {\ssone a}ll{\ssone a}h} (allah)}
  \item {\sstwo a} (ss02) --- provides a vertical stem that experience proves
    to lead readers towards a ``d" as in
    \mbox{{\ssdefault {\sstwo a}orta} (aorta)},
    \mbox{{\ssdefault {\sstwo a}ort{\sstwo a}} (aorta)} or
    \mbox{{\sstwo aero} (aero)}
    until, for instance, a glyph with a taller ascender such as a ``d" or ``b"
    preceeds or succeeds the glyph in question (``a") in order to provide
    the reader with some visual context as evident in
    \mbox{{\sstwo ability} (ability)},
    \mbox{{\sstwo data} (data)},
    \mbox{{\sstwo dragon} (dragon)}. Note that some readers will face
    difficulty reading \mbox{{\sstwo land} (land)} or
    \mbox{{\sstwo poland} (poland)} whereas \mbox{{\sstwo deutschland} (deutschland)} or
    \mbox{{\sstwo wonderland} (wonderland)}
    will be read with minimal effort which may be due to the initial
    characters minimizing the search space for word possibilities and therefore
    introducing a strong bias that depends less on the interpretation of the
    individual letterforms for interpretation.
  \item {\ssthree a} (ss03) --- sloped end is at times still confused for an
    ``o" and may potentially disrupt the flow of a formerly typeset text since
    the width of this character is different and the right bearing has been
    modified to account partially for the amount of negative space above the
    extension
  \item {\ssfour a} (ss04) --- similar to ss03 with the slope of the extension
    inverted leading to a shape that is more reminiscient of ``a"
\end{enumerate}
