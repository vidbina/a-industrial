\documentclass[a4paper]{article}
\usepackage{fontspec, xunicode, xltxtra}
\setlength\parindent{0pt}
%\setmainfont[RawFeature={+liga,+kern,+ss01,+onum},Scale=2.0]{A Industrial Black}
%\setmainfont[Scale=2.0]{A Industrial Black}
\newfontfamily{\ssdefault}[Scale=2.0]{A Industrial Black}
\newfontfamily{\ssone}[StylisticSet=1,Scale=2.0]{A Industrial Black}
\newfontfamily{\sstwo}[StylisticSet=2,Scale=2.0]{A Industrial Black}
\newfontfamily{\ssthree}[StylisticSet=3,Scale=2.0]{A Industrial Black}
\newfontfamily{\ssfour}[StylisticSet=4,Scale=2.0]{A Industrial Black}
\newfontfamily{\ssfive}[StylisticSet=5,Scale=2.0]{A Industrial Black}
\begin{document}
\title{\input{../NAME}}
\author{}
\maketitle
\begin{abstract}
  The different stylistic sets of the font are presented in this preview in order to provide an indication of the alternative glyphs available for use.
  Considering that Industrial classifies as a Display type, there will be plenty of occasions where the default glyphs fail to produce wordshapes that allow for quick reading.
\end{abstract}

%\fontsize{24pt}{36pt}\selectfont

\newpage
\section{default}
{\ssdefault alpha bravo charlie \\
delta echo foxtrot \\
golf hotel india \\
juliet kilo lima \\
mike november oscar \\
papa quebec romeo \\
sierra tango uniform \\
victor whiskey xray \\
yankee zulu \\

\#hashtag \\
@someone \\

41+27=68 \\
39/3=13 \\

}

\newpage
\section{ss01}
{\ssone alpha bravo charlie \\
delta echo foxtrot \\
golf hotel india \\
juliet kilo lima \\
mike november oscar \\
papa quebec romeo \\
sierra tango uniform \\
victor whiskey xray \\
yankee zulu \\

\#hashtag \\
@someone \\

41+27=68 \\
39/3=13 \\

}

\newpage
\section{ss02}
{\sstwo alpha bravo charlie \\
delta echo foxtrot \\
golf hotel india \\
juliet kilo lima \\
mike november oscar \\
papa quebec romeo \\
sierra tango uniform \\
victor whiskey xray \\
yankee zulu \\

\#hashtag \\
@someone \\

41+27=68 \\
39/3=13 \\

}

\newpage
\section{ss03}
{\ssthree alpha bravo charlie \\
delta echo foxtrot \\
golf hotel india \\
juliet kilo lima \\
mike november oscar \\
papa quebec romeo \\
sierra tango uniform \\
victor whiskey xray \\
yankee zulu \\

\#hashtag \\
@someone \\

41+27=68 \\
39/3=13 \\

}

\newpage
\section{ss04}
{\ssfour alpha bravo charlie \\
delta echo foxtrot \\
golf hotel india \\
juliet kilo lima \\
mike november oscar \\
papa quebec romeo \\
sierra tango uniform \\
victor whiskey xray \\
yankee zulu \\

\#hashtag \\
@someone \\

41+27=68 \\
39/3=13 \\

}


\newpage
\section{Cases}
The following sections highlight some challenges with the typeface and proposes
some solutions which have been made available through alternate glyphs through
a number of stylistic sets.

\subsection{a}
The word "area" presented by default as {\ssdefault area} is hardly
distinguishable from {\ssdefault oreo} ("oreo") which is why a typesetter may
decide to use one of the stylistic alternates for "a" to produce
{\ssdefault{\ssone a}re{\ssone a}} (ss01),
{\ssdefault{\ssthree a}re{\ssthree a}} (ss03) or
{\ssdefault{\ssfour a}re{\ssfour a}} (ss04).

The ss02 and ss05 alternatives look more like "dred" or "drod", respectively
{\ssdefault{\sstwo a}re{\sstwo a}} (ss02) or
{\ssdefault{\ssfive a}re{\ssfive a}} (ss05)
which fails to increase legibility.

The different glyphs available for "a" are

\begin{enumerate}
  \item {\ssdefault a}: harmonizes best with the overal design language of the
    typeface but fails in minimizing ambiguity when surrounded by letters with
    a similar x-height and general shape.
  \item {\ssone a} (ss01): provides a horizontal stem that may at times be
    misread as "o".
  \item {\sstwo a} (ss02): provides a vertical stem that experience proves to
    lead readers towards a "d" until, for instance, a "d" or "b" with its
    taller ascender is presented near this version of the "a" in order to
    provide the reader with some visual context e.g. {\sstwo dragon},
    {\sstwo dades} and {\sstwo ability}.
  \item {\ssthree a} (ss03): sloped end is at times still confused for an "o"
    and may potentially disrupt the flow of a formerly typeset text since the width
    of this character is different and the right bearing has been modified to
    account partially for the amount of negative space above the extension.
  \item {\ssfour a} (ss04): similar to ss03 with the slope of the extension
    inverted leading to a shape that is more reminiscient of "a".
\end{enumerate}

\subsection{e}
The word {\ssdefault aeon} ("aeon") is hardly legible because the characters
are all very rectangular by shape as used to represent the character "o"
{\ssdefault o}. In some cases the shape of the word aids in minimizing
ambiguity, a feat we can not really rely on in this case.

By exchanging the "e" for one of the alternatives as in
{\ssdefault a\ssone{e}on} where "e" has been replaced by an alternate from the
first stylistic set (ss01), one may minimize the ambiguity, however; due to the
contraint of the grid the "familiar" wordshape is distrupted, possibly
complicating legibility.

\subsection{x}
{\ssdefault maximum}

\end{document}
