\subsection{x}
Four glyphs are provided for the letter ``x". Where the default glyph provides
an abstracted variant of the ``x" that harmonizes well with the overal design
style of the typeface, alternatives have been provided that increasingly
resemble a classical representation of the letter ``x" with strong diagonal
features as visible in figure \ref{fig:glyphs-x}.

\altfig{x}{3}

\subsubsection{Default {\ssdefault x}}
The default ``x" glyph utilizes a very abstracted design for the representation
of the letter ``x" in order to maintain the rectilinear look of the typeface.

In the case of the word \mbox{{\ssdefault ax} (ax)} this design may not yield
very clear results as the second character could be interpreted as an ``x" or
``I", however; within the context of a more recognizable wordform such as
\mbox{{\ssdefault maximum} (maximum)} or
\mbox{{\ssdefault exile} (exile)} the abstracted representation may yield
acceptable legibility results.

%\mbox{{\ssdefault sexual} (sexual)}
%\mbox{{\ssdefault boxes} (boxes)}
%\mbox{{\ssdefault xenon} (xenon)}

\subsubsection{{\ssone x}.ss01}

In order to minimize ambiguity, some features to emphasize the diagonals of the
``x" are introduced allowing for improved legibility as can be guaged through
the represenation of the word
\mbox{{\ssdefault a{\ssone x}} (ax)},
\mbox{{\ssdefault bo{\ssone x}es} (boxes)},
\mbox{{\ssdefault e{\ssone x}ile} (exile)},
\mbox{{\ssdefault ma{\ssone x}imum} (maximum)}
\mbox{{\ssdefault se{\ssone x}ual} (sexual)} and
\mbox{{\ssdefault {\ssone x}enon} (xenon)}.

The problem with the emphasis of the diagonals is that it somewhat interrupts
the aesthetic flow of the typeface albeit subtle.

\subsubsection{{\sstwo x}.ss02}

The ss02 rendition of the ``x" glyph emphasizes a diagonal features of the x
even more by strong triangular counters which may harmonize less than the ss01
rendition.

The readability does not seem to be a problem as evident in the presentation
of
\mbox{{\ssdefault se{\sstwo x}ual} (sexual)},
\mbox{{\ssdefault bo{\sstwo x}es} (boxes)},
\mbox{{\ssdefault e{\sstwo x}ile} (exile)},
\mbox{{\ssdefault {\sstwo x}enon} (xenon)} and
\mbox{{\ssdefault ma{\sstwo x}imum} (maximum)}.

\subsubsection{{\ssthree x}.ss03}

The ss03 alternative for the ``x" glyph boasts such radical diagonal features
that it doesn't quite fit in with the general typeface, unless it is used in
the middle of a wordform where is serves as an inflection point of sorts as
evident in
\mbox{{\ssdefault se{\ssthree x}ual} (sexual)} and
\mbox{{\ssdefault bo{\ssthree x}es} (boxes)}.
