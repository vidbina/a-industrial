\subsection{a}

The word ``area" presented by default as {\ssdefault area} is hardly
distinguishable from {\ssdefault oreo} (``oreo") which is why a typesetter may
decide to use one of the stylistic alternates for ``a" to produce
{\ssdefault{\ssone a}re{\ssone a}} (ss01),
{\ssdefault{\ssthree a}re{\ssthree a}} (ss03) or
{\ssdefault{\ssfour a}re{\ssfour a}} (ss04).

The ss02 and ss05 alternatives look more like ``dred" or ``drod", respectively
{\ssdefault{\sstwo a}re{\sstwo a}} (ss02) or
{\ssdefault{\ssfive a}re{\ssfive a}} (ss05)
which fails to improve legibility in this particular case.

The following sections discuss the stylistic alternates available for ``a" and
presents some cases where these alternates may prove effective.

\subsubsection{default {\ssdefault a}}

The glyph's double-story design harmonizes well with the overal design
language of the typeface but fails in avoiding ambiguity between letterforms
when this particular glyph is surrounded by letters with a rectangular profile
that happen to be as tall as the x-height as evident in the presentation of the
word \mbox{{\ssdefault area} (area)} which may easily be confused for ``oreo",
``drod'' or ``dred". The arc of the double-story ``a" is not completed due to
the restrictions imposed by the grid that characters are designed unto.

Conversely
\mbox{{\ssdefault train} (train)} or
\mbox{{\ssdefault transport} (transport)}
are rather clearly legible in part due to the distinctive shapes of the words
in question.

\subsubsection{{\ssone a}.ss01}

% TODO: is it a stem or does it go by another name?!?
The ss01 stylistic alternate provides a horizontal stem that may at times be
misread as ``o" as in \mbox{\ssdefault m{\ssone aa}g} which represents ``maag"
but could be misinterpreted as ``moog". The advantage of this glyph is that it
can be used to maintain the horizontal flow of a disinctive wordforms
as evident in
%\mbox{{\ssdefault{\ssone a}rea} (area)},
\mbox{{\ssdefault d{\ssone a}ddy} (daddy)},
\mbox{{\ssdefault {\ssone a}lphabet} (alphabet)},
\mbox{{\ssdefault {\ssone a}lph{\ssone a}bet} (alphabet)},
\mbox{{\ssdefault {\ssone a}llah} (allah)} and
\mbox{{\ssdefault {\ssone a}ll{\ssone a}h} (allah)}.

One point of concern with this rendition of ``a" is that the connection to the
succeeding glyph may need to be ``repaired" through the use of some form of a
ligature as there is not guarantee that separate glyphs will generally be
rendered to produce a seamless connection.

\subsubsection{{\sstwo a}.ss02}

Another attempt at a double story ``a" provides a more pronounced incident of
the arc, but due to the rectiliniear nature of this incident it does not convey
the sense of movement or implied continuation which in many cases incorrectly
leads the readers towards a ``d" as in
% TODO: again verify the usage of the term stem is appropriate in this case
%Instead of a horizontal stem, one may opt for a vertical stem instead. The
%problem with this design is that it leads readers towards a ``d" as in
\mbox{{\ssdefault {\sstwo a}orta} (aorta)},
\mbox{{\ssdefault {\sstwo a}ort{\sstwo a}} (aorta)} or
\mbox{{\sstwo aero} (aero)}
until, for instance, a glyph with a taller ascender such as a ``d" or ``b"
preceeds or succeeds the glyph in question in order to provide the reader with
some visual context as evident in
\mbox{{\sstwo ability} (ability)},
\mbox{{\sstwo data} (data)},
\mbox{{\sstwo dragon} (dragon)}. Note that some readers will face
difficulty reading \mbox{{\sstwo land} (land)} or
\mbox{{\sstwo poland} (poland)} whereas \mbox{{\sstwo deutschland} (deutschland)} or
\mbox{{\sstwo wonderland} (wonderland)}
will be read with minimal effort which may be due to the initial
characters minimizing the search space for word possibilities and therefore
introducing a strong bias that depends less on the interpretation of the
individual letterforms for interpretation of the word as a whole.

\subsubsection{{\ssthree a}.ss03}
sloped end is at times still confused for an
``o" and may potentially disrupt the flow of a formerly typeset text since
the width of this character is different and the right bearing has been
modified to account partially for the amount of negative space above the
extension

\subsubsection{{\ssfour a}.ss04}
similar to ss03 with the slope of the extension
inverted leading to a shape that is more reminiscient of ``a"
